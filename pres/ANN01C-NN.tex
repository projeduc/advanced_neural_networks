% !TEX TS-program = pdflatex
% !TeX program = pdflatex
% !TEX encoding = UTF-8
% !TEX spellcheck = en

\documentclass[xcolor=table]{beamer}

\input{options}

\title[ANN: Neural networks] %
{Advanced neural networks\\Neural networks}  

\changegraphpath{../img/NN/}

\begin{document}

\begin{frame}
\frametitle{Neural networks}
\framesubtitle{Motivation}

\begin{figure}
	\centering
	\hgraphpage[.8\textwidth]{neuron.png}
	\caption{Biological neuron structure \cite{2017-cain}}
\end{figure}

\end{frame}

\begin{frame}
\frametitle{Neural networks}
\framesubtitle{Plan}

\begin{multicols}{2}
	%	\small
	\tableofcontents
\end{multicols}
\end{frame}

\section{Neuron}

\begin{frame}
\frametitle{Neural networks}
\framesubtitle{Neuron}

\begin{itemize}
	\item Let's recall \optword{Logistic regression}
	\item Features are combined linearly: this can be seen as a accumulation of signals
	\[z(x) = \theta_0 + \sum\limits_{j=1}^{N} \theta_j x_j\]
	\item Logistic function is applied on this sum: It is called \keyword{activation function} (send the signal based on a threshold) 
	\[\sigma(z) = \frac{1}{1 + e^{-z}}\] 
	\item To calculate the learning error, a \keyword{cost function} is used
	\item To train te parameters, an \keyword{optimization function} is used
\end{itemize}

\end{frame}

\subsection{Structure}

\begin{frame}
\frametitle{Neural networks: Neuron}
\framesubtitle{Structure}

\hgraphpage[\textwidth]{neuron_structure.pdf}

\end{frame}

\subsection{Activation functions}

\begin{frame}
	\frametitle{Neural networks: Neuron}
	\framesubtitle{Activation functions}

\end{frame}

\begin{frame}
	\frametitle{Neural networks: Neuron}
	\framesubtitle{Activation functions: Sigmoid functions}
	
	\begin{tabular}{ll}
		Logistic function & Hyperbolic tangent \\
		\hgraphpage[.4\textwidth]{logistique.png} & 
		\hgraphpage[.4\textwidth]{tanh.png} \\
		$\color{blue}\sigma(x) = \frac{1}{1 + e^{-x}}$ & 
		$\color{blue}tanh(x) = \frac{e^{x} - e^{-x}}{e^{x} + e^{-x}} = \frac{2}{1 + e^{-2x}} - 1$ \\
		
		$\sigma(x) \in ]0, 1[$ & 
		$tanh(x) \in ]-1, 1[$ \\
		
		$\color{red}\sigma'(x) = \sigma(x) (1-\sigma(x))$ & 
		$\color{red}tanh'(x) = 1 - tanh(x)^2$ \\
	\end{tabular}
	
\end{frame}

\begin{frame}
	\frametitle{Neural networks: Neuron}
	\framesubtitle{Activation functions: Sigmoid functions (2)}
	
	\begin{tabular}{ll}
		Arctangent & Softsign \\
		\hgraphpage[.4\textwidth]{arctan.png} & 
		\hgraphpage[.4\textwidth]{so.png} \\
		$\color{blue}f(x) = \frac{2}{\pi} tan^{-1}(x)$ & 
		$\color{blue}f(x) = \frac{x}{1 + |x|}$ \\
		
		$f(x) \in ]-1, 1[$ & 
		$f(x) \in ]-1, 1[$ \\
		
		$\color{red}f'(x) = \frac{2}{\pi} \frac{1}{x^2 + 1}$ & 
		$\color{red}f'(x) = \frac{1}{(1 + |x|)^2}$ \\
	\end{tabular}
	
\end{frame}

\begin{frame}
	\frametitle{Neural networks: Neuron}
	\framesubtitle{Activation functions: Linear functions}
	
	\begin{tabular}{ll}
		linear function & Identity function \\
		\hgraphpage[.4\textwidth]{lineaire.png} & 
		\hgraphpage[.4\textwidth]{identite.png} \\
		$\color{blue} f(x) = k x$ & 
		$\color{blue} f(x) = x$ \\
		
		$f(x) \in ]-\infty, +\infty[$ & 
		$f(x) \in ]-\infty, +\infty[$ \\
		
		$\color{red}f'(x) = k$ & 
		$\color{red}f'(x) = 1$ \\
	\end{tabular}
	
\end{frame}

\begin{frame}
	\frametitle{Neural networks: Neuron}
	\framesubtitle{Activation functions: Step functions}
	
	\begin{tabular}{ll}
		Heaviside & Sign \\
		\hgraphpage[.4\textwidth]{heaviside.png} & 
		\hgraphpage[.4\textwidth]{signe.png} \\
		$\color{blue} H(x) = \begin{cases}
		0 & \text{si } x < 0 \\
		1 & \text{sinon}
		\end{cases}$ & 
		$\color{blue} S(x) = \begin{cases}
		-1 & \text{si } x < 0 \\
		1 & \text{sinon}
		\end{cases} = 2 H(x) - 1$ \\
		
		$H(x) \in \{0, 1\}$ & 
		$S(x) \in \{-1, 1\}$ \\
		
		$\color{red}H'(x) = \begin{cases}
		0 & \text{si } x \ne 0 \\
		+\infty & \text{sinon}
		\end{cases}$ & 
		$\color{red}S'(x) = 2H'(x)$ \\
	\end{tabular}
	
\end{frame}

\begin{frame}
	\frametitle{Neural networks: Neuron}
	\framesubtitle{Activation functions: Hockey functions}
	
	\begin{tabular}{ll}
		Rectified Linear Unit & Parametric ReLU \\
		\hgraphpage[.4\textwidth]{relu.png} & 
		\hgraphpage[.4\textwidth]{prelu.png} \\
		$\color{blue} ReLU(x) = x H(x) = \max(0, x)$ & 
		$\color{blue} PReLU(\alpha, x) = \begin{cases}
		\alpha x & \text{si } x < 0 \\
		x & \text{sinon}
		\end{cases}$ \\
		
		$ReLU(x) \in [0, +\infty[$ & 
		$PReLU(x) \in ]-\infty, +\infty[$ \\
		
		$\color{red}ReLU'(x) = H(x) = \begin{cases}
		0 & \text{si } x < 0 \\
		1 & \text{sinon}
		\end{cases}$ & 
		$\color{red}PReLU'(x) = \begin{cases}
		\alpha & \text{si } x < 0 \\
		1 & \text{sinon}
		\end{cases}$ \\
	\end{tabular}
	
\end{frame}

\begin{frame}
	\frametitle{Neural networks: Neuron}
	\framesubtitle{Activation functions: Hockey functions (2)}
	
	\begin{tabular}{ll}
		Exponential linear unit & Sigmoid Linear Unit \\
		\hgraphpage[.4\textwidth]{elu.png} & 
		\hgraphpage[.4\textwidth]{slu.png} \\
		$\color{blue} ELU(\alpha, x) = \begin{cases}
		\alpha (e^x - 1) & \text{si } x < 0 \\
		x & \text{sinon}
		\end{cases}$ & 
		$\color{blue} SiLU(x) = x \sigma(x) = \frac{x}{1 + e^{-x}}$ \\
		
		$ELU(\alpha, x) \in ]-\infty, +\infty[$ & 
		$SiLU(x) \in ]-\infty, +\infty[$ \\
		
		$\color{red}ELU'(\alpha, x) = \begin{cases}
		ELU(\alpha, x) + \alpha & \text{si } x < 0 \\
		1 & \text{sinon}
		\end{cases}$ & 
		$\color{red}SiLU'(x) = \sigma(x) (1 + x \sigma(-x))$ \\
	\end{tabular}
	
\end{frame}

\section{Feed Forward}

\begin{frame}
\frametitle{Neural networks}
\framesubtitle{Feed Forward}

\begin{minipage}{0.59\textwidth}
\begin{itemize}
	\item For complex tasks, one neuron is not enough.
	\item For example, \expword{XOR function over two variables}.
	\item If a logistic regression is applied (one neuron with logistic function as activation), the algorithme will try to separate the samples linearly.
	\item any decision line will miss at least one sample.
\end{itemize}
\end{minipage}
\begin{minipage}{0.4\textwidth}
	\hgraphpage[\textwidth]{xor.png}
\end{minipage}

\end{frame}

\subsection{Architecture}

\begin{frame}
	\frametitle{Neural Network: Feed Forward}
	\framesubtitle{Architecture}
	
	\begin{minipage}{0.59\textwidth}
		\begin{itemize}
			\item \optword{Input layer}: information input, \expword{like the dendrites}.
			\item \optword{Hidden layers}: each layer's input is the output of the previous one. 
			It contains a number of neurons. 
			\item \optword{Output layer}: contains the neurons which calculate the output. 
		\end{itemize}
	\end{minipage}
	\begin{minipage}{0.4\textwidth}
		\hgraphpage[\textwidth]{NN.pdf}
	\end{minipage}
	
\end{frame}

\subsection{Input layer}

\begin{frame}
	\frametitle{Neural networks: Feed Forward}
	\framesubtitle{Input layer}
	
\end{frame}

\subsection{Hidden layers}

\begin{frame}
	\frametitle{Neural networks: Feed Forward}
	\framesubtitle{Hidden layers}
	
\end{frame}

\subsection{Output layer}

\begin{frame}
	\frametitle{Neural networks: Feed Forward}
	\framesubtitle{Output layer}
	
\end{frame}


\section{Training}

\begin{frame}
	\frametitle{Neural networks}
	\framesubtitle{Training}
	
\end{frame}

\subsection{Cost functions}

\begin{frame}
	\frametitle{Neural networks: Training}
	\framesubtitle{Cost functions}
	
\end{frame}

\begin{frame}
	\frametitle{Neural networks: Training}
	\framesubtitle{Cost functions: Regression (MSE)}

%\begin{minipage}{0.68\textwidth}
\begin{columns}
	\begin{column}{.68\linewidth}
		\begin{block}{Mean squared error, L2 loss}
			\[MSE = \frac{1}{2} (y - \hat{y})^2\]
			
			\[
			\frac{\partial MSE}{\partial \hat{y}} = \hat{y} - y
			\]
		\end{block}
		
		%\end{minipage}
		%\begin{minipage}{0.3\textwidth}
	\end{column}%
	\begin{column}{.3\linewidth}
		\begin{figure}
			\hgraphpage[\textwidth]{regression-loss_.pdf}
			\caption{Regression cost function \cite{2017-rosenberg}}
		\end{figure}
		%\end{minipage}
	\end{column}
\end{columns}


\end{frame}

\begin{frame}
	\frametitle{Neural networks: Training}
	\framesubtitle{Cost functions: Regression (MSE)}

\begin{itemize}
	\item \optword{Advantages}
	\begin{itemize}
		\item Quadratic function has a unique minimum.
	\end{itemize}
	\item \optword{Disadvantages}
	\begin{itemize}
		\item MSE is less robust to outliers.
	\end{itemize}
\end{itemize}

\end{frame}

\begin{frame}
	\frametitle{Neural networks: Training}
	\framesubtitle{Cost functions: Regression (MAE)}

%\begin{minipage}{0.59\textwidth}
\begin{columns}
	\begin{column}{.68\linewidth}
		\begin{block}{Mean Absolute Error, L1 loss}
			\[MAE = |y - \hat{y}|\]
			
			\[
			\frac{\partial MAE}{\partial \hat{y}} = 
			\begin{cases}
			+1 & \text{ si } \hat{y} > y \\
			-1 & \text{ si } \hat{y} < y \\
			[-1, +1] & \text{ si } \hat{y} = y \\
			\end{cases}
			\]
		\end{block}
	\end{column}%
	\begin{column}{.3\linewidth}
		%\end{minipage}
		%\begin{minipage}{0.4\textwidth}
		\begin{figure}
			\hgraphpage[\textwidth]{regression-loss_.pdf}
			\caption{Cost functions for regression \cite{2017-rosenberg}}
		\end{figure}
		%\end{minipage}
	\end{column}
\end{columns}

\end{frame}

\begin{frame}
	\frametitle{Neural networks: Training}
	\framesubtitle{Cost functions: Regression (MAE)}

\begin{itemize}
	\item \optword{Advantages}
	\begin{itemize}
		\item Less sensitive to outliers.
	\end{itemize}
	\item \optword{Disadvantages}
	\begin{itemize}
		\item The gradient's magnitude does not depend on the error's size; only on the sign of y - ŷ. 
		This leads to the magnitude of the gradient being large even when the error is small.
	\end{itemize}
\end{itemize}

\end{frame}


\begin{frame}
	\frametitle{Neural networks: Training}
	\framesubtitle{Cost functions: Regression (Huber)}

\begin{columns}
\begin{column}{.68\linewidth}
%\begin{minipage}{0.59\textwidth}
\begin{block}{Huber Loss}
	\[J = %\frac{1}{M} \sum\limits_{i=1}^{M} 
	\begin{cases}
	\frac{1}{2} (y - \hat{y})^2 & \text{si } |y - \hat{y}| \le \delta \\
	\delta |y - \hat{y}| - \frac{1}{2} \delta^2 & \text{sinon }\\
	\end{cases}
	\]
	
	\[
	\frac{\partial J}{\partial \hat{y}} = 
	\begin{cases}
	\hat{y} - y & \text{ si } |\hat{y} - y| \le \delta \\
	-\delta & \text{ si } \hat{y} - y < -\delta \\
	+\delta & \text{ si } \hat{y} - y > \delta \\
	\end{cases}
	\]
	
\end{block}

%\end{minipage}
\end{column}%
\begin{column}{.3\linewidth}
%\begin{minipage}{0.4\textwidth}
\begin{figure}
	\hgraphpage[\textwidth]{regression-loss_.pdf}
	\caption{Cost functions for regression \cite{2017-rosenberg}}
\end{figure}
%\end{minipage}
\end{column}
\end{columns}

\end{frame}

\begin{frame}
	\frametitle{Neural networks: Training}
	\framesubtitle{Cost functions: Regression (Huber)}

\begin{itemize}
	\item \optword{Advantages}
	\begin{itemize}
		\item Has both advantages of MSE and MAE.
	\end{itemize}
	\item \optword{Disadvantages}
	\begin{itemize}
		\item Hyper-parameter $\delta$ must be trained.
	\end{itemize}
\end{itemize}

\end{frame}

\begin{frame}
	\frametitle{Neural networks: Training}
	\framesubtitle{Cost functions: Classification (BCE)}

\begin{block}{Binary Cross Entropy Loss}
	\[J = - (y \log(\hat{y}) + (1 - y) \log(1 - \hat{y}))\]
	
	\[
	\frac{\partial J}{\partial \hat{y}} = \frac{\hat{y} - y}{\hat{y} - \hat{y}^2}
	\]
\end{block}

\begin{itemize}
	\item \optword{Advantages}
	\begin{itemize}
		\item Works well with activation functions which model probabilities (sigmoid).
		\item Which means, better probabilities' estimation.
	\end{itemize}
	\item \optword{Disadvantages}
	\begin{itemize}
		\item Linear separation margin is not guaranteed to be optimal.
	\end{itemize}
\end{itemize}

\end{frame}

\begin{frame}
	\frametitle{Neural networks: Training}
	\framesubtitle{Cost functions: Classification (Hinge)}

\begin{block}{Hinge Loss (SVM Loss)}
	\[J = \max(0, 1 - y \hat{y}), y \in \{-1, 1\}\]
	
	\[
	\frac{\partial J}{\partial \hat{y}} = 
	\begin{cases}
	0 & \text{si } y \hat{y} \ge 1 \\
	-y & \text{sinon}
	\end{cases}
	\]
\end{block}

\begin{itemize}
	\item \optword{Advantages}
	\begin{itemize}
		\item Better precision.
	\end{itemize}
	\item \optword{Disadvantages}
	\begin{itemize}
		\item Not smooth and not differentiable; therefore, less optimization algorithms.
		\item Does not directly model the probability $p(y|x)$.
	\end{itemize}
\end{itemize}

\end{frame}

\begin{frame}
	\frametitle{Neural networks: Training}
	\framesubtitle{Cost functions: Multinomial Classification (Cross entropy)}

\begin{block}{Cross Entropy Loss}
	\[J = - \sum\limits_{k=1}^{K} y_{k} \log(\hat{y}_{k}) \text{, où } K \text{ est le nnombre des classes} \]
	
	\[
	\frac{\partial J}{\partial \hat{y}_k} = - \frac{y_{k}}{\hat{y}_{k}}
	\]
\end{block}

%\begin{itemize}
%	\item \optword{Avantages}
%	\begin{itemize}
%		\item 
%	\end{itemize}
%	\item \optword{Inconvénients}
%	\begin{itemize}
%		\item 
%	\end{itemize}
%\end{itemize}

\end{frame}

\subsection{Optimization functions}

\begin{frame}
	\frametitle{Neural networks: Training}
	\framesubtitle{Optimization functions}
	
\end{frame}

\begin{frame}
	\frametitle{Neural networks: Training}
	\framesubtitle{Optimization functions: Gradient descent}

\begin{figure}
	\vgraphpage[.7\textheight]{steepest.png}
	\caption{Gradient descent illustration \cite{2020-calin}}
\end{figure}

\end{frame}

\begin{frame}
	\frametitle{Neural networks: Training}
	\framesubtitle{Optimization functions: Gradient descent}

\begin{algorithm}[H]
	\KwData{$ X, Y, \alpha, T $}
	\KwResult{$ \theta $}
	Initialize $ \theta $ ; $ t = 0 $\;
	\While{$ t < T$ and no convergence}{
		$ \theta = \theta - \alpha \Delta_\theta J(X, Y; \theta) $\;
%		Mettre à jours les $\theta$\;
		t = t + 1 \;
	}
	\caption{Gradient descent}
\end{algorithm}

\end{frame}

\begin{frame}
	\frametitle{Neural networks: Training}
	\framesubtitle{Optimization functions: Gradient descent}

\begin{itemize}
	\item \optword{Properties}: 
	\begin{itemize}
		\item Searches the optimal value in the direction of gradient descent.
		\item Converges with a linear rate.
	\end{itemize}
	\item \optword{Advantages}: 
	\begin{itemize}
		\item When the objective function is convex, the solution is a global optimum.
	\end{itemize}
	\item \optword{Disadvantages}: 
	\begin{itemize}
		\item For each iteration, the gradients of each sample must be calculated. 
		So, the computational cost is very high. 
		\item Training dataset can be too large to be processed in memory.
		\item Can converge to a local minimum when the objective function is not convex.
	\end{itemize}
\end{itemize}

\end{frame}

\begin{frame}
	\frametitle{Neural networks: Training}
	\framesubtitle{Optimization functions: Stochastic gradient descent}

\begin{algorithm}[H]
	\KwData{$ X, Y, \alpha, T$}
	\KwResult{$ \theta $}
	Initialize $ \theta $ ; $ t = 0 $\;
	\While{$ t < T$ and no convergence}{
		Randomly choose $ (X^{(i)}, Y^{(i)}) \in (X, Y)$\;
		$ \theta = \theta - \alpha \Delta_\theta J(X^{(i)}, Y^{(i)}; \theta) $\;
%		Mélanger les données aléatoirement\;
%		\ForEach{$ (X^{(i)}, Y^{(i)}) \in (X, Y)$}{
%			$ \theta = \theta - \alpha \Delta_\theta J(X^{(i)}, Y^{(i)}; \theta) $\;
%%			Mettre à jours les $\theta$\;
%		}
		t = t + 1 \;
	}
	\caption{Stochastic gradient descent}
\end{algorithm}

\end{frame}

\begin{frame}
	\frametitle{Neural networks: Training}
	\framesubtitle{Optimization functions: Stochastic gradient descent}

\begin{itemize}
\item \optword{Properties}: 
\begin{itemize}
	\item Parameters are separately updated for each sample.
\end{itemize}
\item \optword{Advantages}: 
\begin{itemize}
	\item Can avoid local minima.
\end{itemize}
\item \optword{Disadvantages}: 
\begin{itemize}
	\item It is difficult for the model to converge to a minimum.
\end{itemize}
\end{itemize}

\end{frame}


\begin{frame}
	\frametitle{Neural networks: Training}
	\framesubtitle{Optimization functions: Mini-Batch gradient descent}

\begin{algorithm}[H]
	\KwData{$ X, Y, \alpha, T, b $}
	\KwResult{$ \theta $}
	initialiser $ \theta $ ; $ t = 0 $\;
	\While{$ t < T$ and no convergence}{
		Randomly mix data\;
		Divide data to $b$ batches\;
		\ForEach{$ (X^b, Y^b) \in batches$}{
			$ \theta = \theta - \alpha \Delta_\theta J(X^b, Y^b; \theta) $\;
%			Mettre à jours les $\theta$\;
		}
		\tcp{Parameters can be updated using the average of the steps}
		t = t + 1 \;
	}
	\caption{Mini-Batch gradient descent}
\end{algorithm}

\end{frame}

\begin{frame}
	\frametitle{Neural networks: Training}
	\framesubtitle{Optimization functions: Mini-Batch gradient descent}

\begin{itemize}
	\item \optword{Proprieties}: 
	\begin{itemize}
		\item Training is done on batches (subsets) of the dataset.
		\item Using a hyper-parameter $b$ as the size of a batch.
	\end{itemize}
	\item \optword{Advantages}: 
	\begin{itemize}
		\item May avoid local minimum due to frequent updates.
		\item Efficient with large training data.
	\end{itemize}
	\item \optword{Disadvantages}: 
	\begin{itemize}
		\item Hyper-parameter $b$ must be configured.
	\end{itemize}
\end{itemize}

\end{frame}


\begin{frame}
	\frametitle{Neural networks: Training}
	\framesubtitle{Optimization functions: Adaptative gradient descent (AdaGrad)}

\begin{algorithm}[H]
	\KwData{$ X, Y, \alpha, T$}
	\KwResult{$ \theta $}
	Initialize $ \theta $ ; $ t = 0 $\;
	Initialize $v$ ($|v| = |\theta|$) to zero\;
	\While{$ t < T$ and no convergence}{
		$ v = v + (\Delta_\theta J(X, Y; \theta))^2 $\;
		$ \theta = \theta - \frac{\alpha}{\sqrt{v + \epsilon}} \Delta_\theta J(X, Y; \theta) $\;
%		Mettre à jours les $\theta$\;
		\tcp{epsilon is used to avoid division by 0, usually 1e-8}
		t = t + 1 \;
	}
	\caption{AdaGrad \cite{2011-duchi-al}}
\end{algorithm}

\end{frame}

\begin{frame}
	\frametitle{Neural networks: Training}
	\framesubtitle{Optimization functions: Adaptative gradient descent (AdaGrad)}

\begin{itemize}
	\item \optword{Proprieties}: 
	\begin{itemize}
		\item Learning rate is adaptively adjusted based on squares' sum of all past gradients.
	\end{itemize}
	\item \optword{Advantages}: 
	\begin{itemize}
		\item Learning rate of each parameter is adaptively adjusted.
	\end{itemize}
	\item \optword{Disadvantages}: 
	\begin{itemize}
		\item Over time, the learning rate drops to zero; the algorithm is not desirable for non-convex functions.
	\end{itemize}
\end{itemize}

\end{frame}

\begin{frame}
	\frametitle{Neural networks: Training}
	\framesubtitle{Optimization functions: Root Mean Square Propagation (RMSProp)}

\begin{algorithm}[H]
	\KwData{$ X, Y, \alpha, T, \beta$}
	\KwResult{$ \theta $}
	Initialize $ \theta $ ; $ t = 0 $\;
	Initialize $v$ ($|v| = |\theta|$) to zero\;
	\While{$ t < T$ and no convergence}{
		$ v = \beta v + (1-\beta)(\Delta_\theta J(X, Y; \theta))^2 $\;
		$ \theta = \theta - \frac{\alpha}{\sqrt{v} + \epsilon} \Delta_\theta J(X, Y; \theta) $\;
%		Mettre à jours les $\theta$\;
		t = t + 1 \;
	}
	\caption{RMSProp}
\end{algorithm}

\end{frame}

\begin{frame}
	\frametitle{Neural networks: Training}
	\framesubtitle{Optimization functions: Root Mean Square Propagation (RMSProp)}

\begin{itemize}
	\item \optword{Proprieties}: 
	\begin{itemize}
		\item Using exponential moving average.
		\item By default, $\beta = 0.9$ 
	\end{itemize}
	\item \optword{Advantages}: 
	\begin{itemize}
		\item Improves inefficient learning problem in the advanced stages of AdaGrad.
		\item Suitable for optimization of non-stationary and non-convex problems.
	\end{itemize}
	\item \optword{Disadvantages}: 
	\begin{itemize}
		\item At the end of training, the updating process can be repeated around the local minimum.
	\end{itemize}
\end{itemize}

\end{frame}

\begin{frame}
	\frametitle{Neural networks: Training}
	\framesubtitle{Optimization functions: Adaptive Moment Estimation (Adam)}

\begin{algorithm}[H]
	\KwData{$ X, Y, \alpha, T, \beta_1, \beta_2$}
	\KwResult{$ \theta $}
	Initialize $ \theta $ ; $ t = 0 $\;
	Initialize $v$ et $m$ ($|v| = |m| = |\theta|$) to zero\;
	\While{$ t < T$ and no convergence}{
		$ g = \Delta_\theta J(X, Y; \theta) $\;
		$ m = \beta_1 m + (1-\beta_1) g $; $\hat{m} = \frac{m}{1-\beta_1^t}$\;
		$ v = \beta_2 v + (1-\beta_2) g^2 $;
		$\hat{v} = \frac{v}{1-\beta_2^t}$\;
		$ \theta = \theta - \frac{\alpha \hat{m}}{\sqrt{\hat{v}} + \epsilon} $\;
%		Mettre à jours les $\theta$\;
		t = t + 1 \;
	}
	\caption{Adam \cite{2015-kingma-ba}}
\end{algorithm}

\end{frame}

\begin{frame}
	\frametitle{Neural networks: Training}
	\framesubtitle{Optimization functions: Adaptive Moment Estimation (Adam)}

\begin{itemize}
	\item \optword{Proprieties}: 
	\begin{itemize}
		\item Using the first and second order moments to dynamically adjust the learning rate of each parameter.
		\item Adds bias correction.
		\item By default, $\beta_1 = 0.9, \beta_2 = 0.999$ 
	\end{itemize}
	\item \optword{Advantages}: 
	\begin{itemize}
		\item Gradient descent process is relatively stable.
		\item Suitable for most non-convex optimization problems with large datasets and high dimensional space.
	\end{itemize}
	\item \optword{Disadvantages}: 
	\begin{itemize}
		\item May not converge in some cases.
	\end{itemize}
\end{itemize}

\end{frame}

\subsection{Back-propagation}

\begin{frame}
	\frametitle{Neural networks: Training}
	\framesubtitle{Back-propagation}
	
	\hgraphpage[\textwidth]{RNPA.pdf}
	
\end{frame}

\begin{frame}
	\frametitle{Neural networks: Training}
	\framesubtitle{Back-propagation: Example}
	
	\hgraphpage[\textwidth]{RNPA-exp.pdf}
	
\end{frame}

\begin{frame}
	\frametitle{Neural networks: Training}
	\framesubtitle{Back-propagation: Example (1)}
	
	%\begin{itemize}
	%	\item On essaye de mettre à jour $w_{11}^{(3)}$
	%\end{itemize}
	Update $w_{11}^{(4)}$
	
	$ 
	\frac{\partial J}{\partial w_{11}^{(4)}} = \overbrace{\frac{\partial J}{\partial f_{1}^{(4)}} \frac{\partial f_{1}^{(4)}}{\partial z_{1}^{(4)}}}^{\delta_{1}^{(4)}} \overbrace{\frac{\partial z_{1}^{(4)}}{\partial w_{11}^{(4)}}}^{a_{1}^{(3)}}
	$
	
	$ 
	\frac{\partial J}{\partial f_{1}^{(4)}} = \frac{(0.840, 0.843) - (0, 1)}{(0.840, 0.843) - (0.840, 0.843)^2} 
	= (6.25, -1.186)
	$
	
	$ 
	\frac{\partial f_{1}^{(4)}}{\partial z_{1}^{(4)}} = (0.840, 0.843) (0.160, 0.157) = (0.134, 0.132)
	$
	
	$
	\delta_{1}^{(4)} = (6.25, -1.186) (0.134, 0.132) \approx (0.838, -0.157)
	$
	
	$
	\frac{\partial J}{\partial w_{11}^{(4)}} = moy((0.838, -0.157) (0.555, 0.612)) 
	\approx moy(0.465, -0.096) = 0.184
	$
	
	$
	w_{11}^{(4)} = 0.7 - 1 * (0.184) = 0.516 \text{ , when } \alpha = 1
	$
	
\end{frame}

\begin{frame}
	\frametitle{Neural networks: Training}
	\framesubtitle{Back-propagation: Example (2)}
	
	%\begin{itemize}
	%	\item On essaye de mettre à jour $w_{11}^{(3)}$
	%\end{itemize}
	Update $w_{21}^{(4)}$
	
	$ 
	\frac{\partial J}{\partial w_{21}^{(4)}} = \overbrace{\frac{\partial J}{\partial f_{1}^{(4)}} \frac{\partial f_{1}^{(4)}}{\partial z_{1}^{(4)}}}^{\delta_{1}^{(4)}} \overbrace{\frac{\partial z_{1}^{(4)}}{\partial w_{21}^{(4)}}}^{a_{2}^{(3)}}
	$
	
	
	$
	\delta_{1}^{(4)} = (0.838, -0.157)
	$
	
	$
	\frac{\partial J}{\partial w_{21}^{(4)}} = moy((0.838, -0.157) (0.386, 0.360)) 
	\approx moy(0.323, -0.056) = 0.134
	$
	
	$
	w_{21}^{(4)} = 0.7 - 1 * (0.134) = 0.566 \text{ , when } \alpha = 1
	$
	
\end{frame}

\begin{frame}
	\frametitle{Neural networks: Training}
	\framesubtitle{Back-propagation: Example (3)}
	
	Update $w_{11}^{(3)}$
	
	$
	\frac{\partial J}{\partial w_{11}^{(3)}} = 
	\overbrace{
		\overbrace{
			\frac{\partial J}{\partial f_{1}^{(4)}} 
			\frac{\partial f_{1}^{(4)}}{\partial z_{1}^{(4)}}
		}^{\delta_{1}^{(4)}} 
		\overbrace{
			\frac{\partial z_{1}^{(4)}}{\partial f_{1}^{(3)}}
		}^{w_{11}^{(4)}} 
		\frac{\partial f_{1}^{(3)}}{\partial z_{1}^{(3)}} 
	}^{\delta_{1}^{(3)}} 
	\overbrace{
		\frac{\partial z_{1}^{(3)}}{\partial w_{11}^{(3)}}
	}^{a_{1}^{(2)}}
	\text{ Ici, on utilise l'ancien } w_{11}^{(4)}
	$
	
	$
	\frac{\partial f_{1}^{(3)}}{\partial z_{1}^{(3)}} \approx 
	(0.555, 0.612) (0.445, 0.388) = (0.247, 0.237)
	$
	
	$
	\delta_{1}^{(3)} = (0.838, -0.157) * 0.7 * (0.247, 0.237) \approx (0.145, -0.026)
	$
	
	$
	\frac{\partial J}{\partial w_{11}^{(2)}} = moy((0.145, -0.026) (0.622, 0.900)) 
	= moy(0.090, -0.023) = 0.033
	$
	
	$
	w_{11}^{(2)} = 0.3 - 1 * (0.033) = 0.267 \text{ , when } \alpha = 1
	$
	
\end{frame}


\begin{frame}
	\frametitle{Neural networks: Training}
	\framesubtitle{Back-propagation: Example (4)}
	
	Update $w_{12}^{(3)}$
	
	$
	\frac{\partial J}{\partial w_{12}^{(3)}} = 
	\overbrace{
		\overbrace{
			\frac{\partial J}{\partial f_{1}^{(4)}} 
			\frac{\partial f_{1}^{(4)}}{\partial z_{1}^{(4)}}
		}^{\delta_{1}^{(4)}} 
		\overbrace{
			\frac{\partial z_{1}^{(4)}}{\partial f_{2}^{(3)}}
		}^{w_{11}^{(4)}} 
		\frac{\partial f_{2}^{(3)}}{\partial z_{2}^{(3)}} 
	}^{\delta_{2}^{(3)}} 
	\overbrace{
		\frac{\partial z_{2}^{(3)}}{\partial w_{12}^{(3)}}
	}^{a_{1}^{(2)}}
	\text{ Ici, on utilise l'ancien } w_{12}^{(4)}
	$
	
	$
	\frac{\partial f_{2}^{(3)}}{\partial z_{2}^{(3)}} \approx 
	(0.386, 0.360) (0.614, 0.64) = (0.237, 0.230)
	$
	
	$
	\delta_{2}^{(3)} = (0.838, -0.157) * 0.7 * (0.237, 0.230) \approx (0.139, -0.025)
	$
	
	$
	\frac{\partial J}{\partial w_{11}^{(2)}} = moy((0.139, -0.025) (0.622, 0.900)) 
	= moy(0.086, -0.023) = 0.032
	$
	
	$
	w_{11}^{(2)} = - 0.1 - 1 * (0.032) = - 0.132 \text{ , when } \alpha = 1
	$
	
\end{frame}


\begin{frame}
	\frametitle{Neural networks: Training}
	\framesubtitle{Back-propagation: Example (5)}
	
	Update $w_{11}^{(2)}$
	
	$
	\frac{\partial J}{\partial w_{11}^{(2)}} = 
	\overbrace{ 
		\left(
		\delta_{1}^{(3)} 
		\overbrace{
			\frac{\partial z_{1}^{(3)}}{\partial f_{1}^{(2)}}
		}^{w_{11}^{(3)}}
		+
		\delta_{2}^{(3)} 
		\overbrace{
			\frac{\partial z_{2}^{(3)}}{\partial f_{1}^{(2)}}
		}^{w_{12}^{(3)}}
		\right)
		\frac{\partial f_{1}^{(2)}}{\partial z_{1}^{(2)}}
	}^{\delta_{1}^{(2)}} 
	\overbrace{
		\frac{\partial z_{1}^{(2)}}{\partial w_{11}^{(2)}}
	}^{a_{1}^{(1)}}
	$
	
	$
	\frac{\partial f_{1}^{(2)}}{\partial z_{1}^{(2)}} = (0.622, 0.900) (0.378, 0.100) = (0.235, 0.09)
	$
	
	$
	\delta_{1}^{(2)} = \left((0.145, -0.026) * (0.3) + (0.139, -0.025) * (-0.1)\right) * (0.235, 0.09) \approx (0.039, -0.01)
	$
	
	
	%$
	%\frac{\partial J}{\partial w_{11}^{(2)}} = moy((0.039, -0.01) (2, 3)) 
	%= moy(0.078, -0.03) = 0.024
	%$
	%
	%$
	%w_{11}^{(2)} = 0.5 - 1 * (0.024) = 0.476 \text{ , supossant } \alpha = 1
	%$
	
\end{frame}

\begin{frame}
	\frametitle{Neural networks: Training}
	\framesubtitle{Back-propagation: General case}
	
	\begin{itemize}
		\item Calculate $\delta^{(l)}$ where $l$ is the layer's number
	\end{itemize}
	
	\[ 
	\delta^{(sortie)} = 
	\frac{\partial J}{\partial f^{(sortie)}} \frac{\partial f^{(sortie)}}{\partial z^{(sortie)}}
	\,,\,
	\delta^{(l)} = \frac{\partial f^{(l)}}{\partial z^{(l)}} w^{(l+1)} \delta^{(l+1)}
	\]
	
	\begin{itemize}
		\item Calculate gradients
	\end{itemize}
	
	\[ 
	\frac{\partial J}{\partial w^{(l)}} = a^{(l-1)} \delta^{(l)}
	\,,\,
	\frac{\partial J}{\partial b^{(l)}} = \delta^{(l)}
	\]
	
	\begin{itemize}
		\item Update parameters
	\end{itemize}
	
	\[ 
	w = w - \alpha \frac{\partial J}{\partial w^{(l)}}
	\,,\,
	b = b - \alpha \frac{\partial J}{\partial b^{(l)}}
	\]
	
	
\end{frame}

\begin{frame}
	\frametitle{Neural networks: Training}
	\framesubtitle{Back-propagation: A little fun}
	
	\hgraphpage[.58\textwidth]{humour-retro1.jpeg}
	\hgraphpage[.4\textwidth]{humour-retro2.jpg}
	
	
\end{frame}

\subsection{Regularization}

\begin{frame}
	\frametitle{Neural networks}
	\framesubtitle{Regularization}

	
\end{frame}

\begin{frame}
	\frametitle{Neural networks: Regularization}
	\framesubtitle{Dropout layer}
	
	\begin{itemize}
		\item Deactivate the connections temporarily and randomly.
		\item Dropout layer has a hyper-parameter: drop rate.
		\item Deactivates some outputs (considers them as 0) in training (not inference).
	\end{itemize}
	
\end{frame}


\begin{frame}
	\frametitle{Neural networks: Regularization}
	\framesubtitle{Dropout layer: A little fun}
	
	\hgraphpage[.35\textwidth]{humour-overfiting2.jpeg}
	\hgraphpage[.63\textwidth]{humour-overfiting1.png}
	
\end{frame}


\insertbibliography{ANN01-NN}{*}




\end{document}